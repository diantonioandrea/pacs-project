Having built a mesh over a polygonal domain, the Poisson problem can be solved by first constructing the problem's matrix $\MA$ and the forcing term $\VB$ as described in \eqref{matrix} and \eqref{forcing}.

The \lstinline{laplacian} function constructs the matrices used for solving the problem and evaluating errors by computing all terms in \eqref{boa} for each element. The resulting matrices are in sparse form.

The \lstinline{forcing} function constructs the forcing term by evaluating \eqref{forcing} and enforces the Dirichlet boundary condition \eqref{dirichlet} through penalization.

\cite{Saad2003} The solution to the matrix equation $\MA \Vector{\upsilon}^k_h = \VB$ is obtained using the \lstinline{BICGSTAB} algorithm, with the \lstinline{GMRES} algorithm used if the first one fails to converge within a fixed number of steps. Both of these iterative algorithms for sparse matrices have been implemented in the \textit{algebra} section of the code.

On adaptively refined meshes, $\kappa(\MA)$ rapidly grows, necessitating the use of a preconditioner within the custom solver \lstinline{lapsolver}.

Let $\MM$ and $(\mathcal{V}_{DG} + \mathcal{S}_{DG})$ be the mass and $DG$ matrices. The $L^2$ and $DG$ errors are then evaluated by first computing the modal coefficients $\Vector{\upsilon}_m$ for the exact solution $u$ and solving $\MM \Vector{u} = \Vector{\upsilon}_m$ using the \lstinline{DB}\footnote{Block diagonal algorithm.} algorithm. Thus, the error vector is $\Vector{e} = \Vector{\upsilon} - \Vector{\upsilon}^k_h$. Hence:

\begin{gather}
    \lVert u - u^k_h \rVert_{\LT(\Omega)} = \sqrt{\Vector{e}^\intercal \MM \Vector{e}}, \\
    \lVert u - u^k_h \rVert_{DG} = \sqrt{\Vector{e}^\intercal (\mathcal{V}_{DG} + \mathcal{S}_{DG}) \Vector{e}},
\end{gather}

Some examples are provided in the following sections.

\newpage
\subsection{A code snippet}

Here's a snippet to illustrate the Poisson solution process from the user's perspective:

\lstinputlisting[style=cpp, firstline=11]{../snippets/poisson.cpp}
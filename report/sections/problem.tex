\subsection{The Poisson problem}

This problem is chosen for its simplicity, which allows us to focus on the formulation and implementation of the \textit{hp-Adaptive} Discontinuous Galërkin (DG) method without the complexities introduced by more intricate problems. The Poisson equation is a fundamental elliptic partial differential equation that arises in various fields, such as electrostatics, heat transfer, and fluid dynamics. Its relatively simple form makes it an ideal candidate for testing numerical methods like the DG method, especially when exploring advanced features such as hp-adaptivity.

Consider the domain $\Omega \subset \mathbb{R}^2$. The goal is to find $u \in C^2(\Omega)$ such that, for any source function $f \in C(\Omega)$ and boundary condition $g \in C^2(\partial \Omega)$, the following strong form of the Poisson equation holds:

\begin{gather}
    \begin{cases} \label{strong_stokes}
        - \Delta u = f & \Forall \Vector{x} \in \Omega, \\
        u = g & \Forall \Vector{x} \in \partial \Omega.
    \end{cases}
\end{gather}

\subsection{The weak formulation}

The goal is to find $u \in \{v \in V : v \vert_{\partial \Omega} = g \}$ such that, for any $f \in V^*$, the following equation is satisfied:

\begin{gather} \label{weak_stokes}
    \boa{u}{v} = \langle f, v \rangle \quad \Forall v \in V,
\end{gather}

where $\boa{\cdot}{\cdot}: V \times V \rightarrow \mathbb{R}$ is a bilinear form defined as follows:

\begin{gather}
    \boa{u}{v} = \int_{\Omega} \grad u \cdot \grad v \, d\omega, \label{a}
\end{gather}

This weak formulation serves as the foundation for numerical approximation methods, such as the Discontinuous Galërkin method.

\subsection{Interior penalty DG method}

The Interior Penalty DG method is chosen for solving elliptic partial differential equations like the Poisson problem due to its flexibility in handling complex geometries and its capability to manage discontinuities in the solution. Specifically, a symmetric interior penalty method is employed, which balances stability and accuracy by introducing penalty terms to enforce continuity between elements. This symmetric variant ensures coercivity and consistency, leading to improved stability properties.

Consider a symmetric interior penalty method for this problem, where the objective is to find $ u^k_h \in V^k_h $ such that:

\begin{gather}
    \boa{u^k_h}{v^k_h} = \langle f, v^k_h \rangle \quad \Forall v^k_h \in V^k_h.
\end{gather}

Let $\{\Tau_h\}_h$ denote a sequence of polygonal meshes and define $ V^k_h $ as follows:

\begin{gather}
    V^k_h = \left\{ v^k_h \in \LT(\Omega): v^k_h \vert_K \in \PK{k}(K) ~ \Forall K \in \Tau_h \right\},
\end{gather}

The bilinear form $\boa{\cdot}{\cdot}$ includes terms to account for discontinuities between elements, defined as:

\begin{align} 
    \begin{split} \label{boa}
        \boa{v^k_h}{w^k_h} &= \sum_{K \in \Tau_h} \int_K \grad v^k_h \cdot \grad w^k_h \, d \omega \\
        &- \sum_{F \in \F} \int_F \{\!\!\{ \grad w^k_h \}\!\!\} \cdot \llbracket v^k_h \rrbracket \, d \sigma  \\
        &- \sum_{F \in \F} \int_F \llbracket w^k_h \rrbracket \cdot \{\!\!\{ \grad v^k_h \}\!\!\} \, d \sigma \\
        &+ \sum_{F \in \F} \int_F \gamma \llbracket w^k_h \rrbracket \cdot \llbracket v^k_h \rrbracket \, d \sigma \quad \Forall v^k_h, w^k_h \in V^k_h.
    \end{split}
\end{align}

Here, $\llbracket \cdot \rrbracket$ and $\{\!\!\{\cdot\}\!\!\}$ represent the jump and average operators, respectively, used to handle discontinuities across element boundaries. The penalty parameter $\gamma$ is introduced to penalize discontinuities, thus ensuring stability.

Dirichlet boundary conditions are enforced by penalization:

\begin{gather} \label{dirichlet}
    \llbracket u \rrbracket = (u - g) \Vector{n} \quad \Forall F \in \F : F \cap \partial \Omega \neq \emptyset.
\end{gather}

This approach ensures a consistent treatment of boundary conditions within the DG framework.

\subsection{Polynomial basis}

Let $\left\{ \phi_i \right\}_{i = 1}^N$ denote a basis for the space $V^k_h$. The approximate solution can be expressed as:

\begin{gather} \label{decomposition}
    u^k_h = \sum_{i = 1}^N \upsilon_i \phi_i \quad \Forall u^k_h \in V^k_h,
\end{gather}

where $\left\{\phi_i\right\}$ are the basis functions and $\left\{\upsilon_i\right\}$ are the coefficients to be determined. 

The goal is to find $\Vector{\upsilon} \in \mathbb{R}^N$ such that:

\begin{gather}
    \MA \Vector{\upsilon} = \VB,
\end{gather}

where $\MA \in \mathbb{R}^{N \times N}$ and $\VB \in \mathbb{R}^N$ are defined as:

\begin{align}
    \MA_{ij} &= \boa{\phi_i}{\phi_j}, \label{matrix} \\ 
    \VB_i &= \langle f, \phi_i \rangle. \label{forcing}
\end{align}

Here, $\MA_{ij}$ represents the matrix of bilinear forms computed using the basis functions $\phi_i$ and $\phi_j$, while $\VB_i$ denotes the vector of the actions of the linear functional $f$ on the basis functions $\phi_i$.

Legendre polynomials are used as the basis functions for $V^k_h$ due to their orthogonality and numerical properties.
\subsection{Estimating the decay rate of the Legendre coefficients}

Having tested the \textit{h-adaptive} refinement capabilities for this $DG$ implementation, the next and final step is to implement \textit{p-adaptive} refinement using a test of analyticity.

\cite{Eibner2007} Since the solution is represented by the coefficients of Legendre polynomials, analyticity can be assessed by evaluating their rate of decay.

Given \eqref{decomposition}, the following relation can be written for every element $K$:

\begin{gather}
    u^{k, ij}_{h, K} = c_{ij} \phi_{ij} \quad \Forall i, j : i + j = k.
\end{gather}

Assuming smoothness for $u^k_{h, K}$, the following holds:

\begin{gather}
    \Exists a_K, b_K \in \R : c_{ij} \approx a_K e^{-b_K (i + j)}.
\end{gather}

An estimate of $b_K$ through a linear fit of $\log(\lvert c_{ij} \rvert)$ is the key to choosing \textit{p-refinement} over \textit{h-refinement}. In fact, $u^k_{h, K}$ is said to be smooth if $b_K$ exceeds a certain threshold, fixed to $1.5$ in the following numerical tests.

The marking strategy is slightly modified so that the elements $K$ to be refined are chosen in the following way:

\begin{gather}
    \eta_K^2 > \sigma \bar{\eta}^2,
\end{gather}

where:

\begin{gather}
    \bar{\eta}^2 = \frac{1}{\lvert \Tau \rvert} \sum_{K \in \Tau} \eta_K^2.
\end{gather}

\newpage
\subsubsection{Errors}

\cite{Eibner2007} The following plot confirms the expected convergence rate over the L-shaped domain, up to some points, and over the square domain for a smooth, albeit pathological, function:

\begin{gather}
    \lVert u - u^k_h \rVert_{DG, \, \text{Square}} \approx a \, e^{-b \, \text{DOFs}^{1/2}}, \\
    \lVert u - u^k_h \rVert_{DG, \, \text{L-shape}} \approx a \, e^{-b \, \text{DOFs}^{1/3}}.
\end{gather}

\begin{figure}[!ht]
    \begin{subfigure}[b]{0.45\textwidth}
		% HP v DOFs template for TikZ.

\begin{tikzpicture}
\begin{axis}[
    xlabel={$DOFs^{1/3}$},
    legend pos=north east,
    ymode=log
]

\addplot[solarized-base02, mark=*] coordinates {(15.811388300841896,7.96311) (20.85665361461421,6.46708) (25.592967784139454,2.88996) (35.14256678161116,0.972308) (37.76241517699841,0.28022) (39.08964057138413,0.129849) (44.181444068749045,0.083698) (53.786615435440815,0.0436296) (62.120849961989414,0.0214149) (72.29799443968,0.00874589) (85.28188553262645,0.00475347) (100.53854982045445,nan)};
\addlegendentry{$DG$ Error}

\end{axis}
\end{tikzpicture}
	\end{subfigure}
	\hfill
	\begin{subfigure}[b]{0.45\textwidth}
		% HP v DOFs template for TikZ.

\begin{tikzpicture}
\begin{axis}[
    xlabel={$DOFs^{1/2}$}, % Edit if needed.
    legend pos=north east,
    ymode=log
]

\addplot[solarized-base02, mark=*] coordinates {(35.35533905932738,5.22824) (36.05551275463989,3.08808) (39.585350825778974,1.40074) (40.19950248448356,0.568479) (42.68489194082609,0.184123) (43.76071297408213,0.0545945) (45.12205669071391,0.0267047) (46.69047011971501,0.0206007) (59.51470406546604,0.0118629) (70.08566187174092,0.00542351) (76.6355009117837,0.0022026) (78.20485918406861,0.000978576) (81.65169930871004,0.000405115)};
\addlegendentry{$DG$ Error}

\end{axis}
\end{tikzpicture}
	\end{subfigure}
    \caption{DG error versus $DOFs^{1/2}$ on a sequence of \textit{hp-adaptively} refined meshes over a square domain. $k_0 = 3$, $N_0 = 25$ (left) and $N_0 = 125$ (right).}
\end{figure}

\begin{figure}[!ht]
    \begin{subfigure}[b]{0.45\textwidth}
		% HP v DOFs template for TikZ.

\begin{tikzpicture}
\begin{axis}[
    xlabel={$DOFs^{1/3}$},
    legend pos=north east,
    ymode=log
]

\addplot[solarized-base02, mark=*] coordinates {(6.299605249474365,0.220458) (6.46330407009565,0.213463) (7.230426792525689,0.147802) (8.434326653017491,0.0864727) (9.205164082515887,0.063222) (9.86484829732188,0.0473818) (10.446439268223186,0.0367468) (10.969613104865235,0.0304151) (11.447142425533317,0.0274114) (11.974482814936371,0.0241808) (13.103706971044481,0.0156641)};
\addlegendentry{$DG$ Error}

\end{axis}
\end{tikzpicture}
	\end{subfigure}
	\hfill
	\begin{subfigure}[b]{0.45\textwidth}
		% HP v DOFs template for TikZ.

\begin{tikzpicture}
\begin{axis}[
    xlabel={$DOFs^{1/3}$},
    legend pos=north east,
    ymode=log
]

\addplot[solarized-base02, mark=*] coordinates {(10.772173450159418,0.119546) (10.899918636993316,0.114459) (11.40629589648136,0.0620643) (11.807100963907336,0.0391474) (12.182398697839568,0.0255464) (12.523157073504041,0.0172725) (12.83418595935584,0.0121515) (13.169381127751542,0.00902302) (13.491993423410849,0.0074057) (14.325697513313383,0.00665863) (15.149613862436668,0.00583916) (16.103495762398754,0.00311361) (18.31344031405736,0.00255002)};
\addlegendentry{$DG$ Error}

\end{axis}
\end{tikzpicture}
	\end{subfigure}
    \caption{DG error versus $DOFs^{1/3}$ on a sequence of \textit{hp-adaptively} refined meshes over a L-shaped domain. $k_0 = 3$, $N_0 = 25$ (left) and $N_0 = 125$ (right).}
\end{figure}

\newpage
\subsubsection{Meshes}

\begin{figure}[!ht]
	\centering
    \includegraphics[width=5.5cm]{meshes/adaptive/square_hp_2.pdf}
	\includegraphics[width=5.5cm]{meshes/adaptive/square_hp_5.pdf}
	\includegraphics[width=5.5cm]{meshes/adaptive/square_hp_10.pdf}
    \includegraphics[width=5.5cm]{meshes/adaptive/square_hp_125_2.pdf}
	\includegraphics[width=5.5cm]{meshes/adaptive/square_hp_125_5.pdf}
	% \includegraphics[width=5.5cm]{meshes/adaptive/square_hp_125_10.pdf}
	\caption{Square mesh after 2, 5 and 10 refinements, $N_0 = 25$ (top) and $N_0 = 125$ (bottom).}
\end{figure}

\begin{figure}[!ht]
	\centering
	\includegraphics[width=5.5cm]{meshes/adaptive/lshape_hp_5.pdf}
	\includegraphics[width=5.5cm]{meshes/adaptive/lshape_hp_10.pdf}
	\includegraphics[width=5.5cm]{meshes/adaptive/lshape_hp_15.pdf}
    \includegraphics[width=5.5cm]{meshes/adaptive/lshape_hp_125_5.pdf}
	\includegraphics[width=5.5cm]{meshes/adaptive/lshape_hp_125_10.pdf}
	% \includegraphics[width=5.5cm]{meshes/adaptive/lshape_hp_125_15.pdf}
	\caption{L-shaped mesh after 5, 10 and 15 refinements, $N_0 = 25$ (top) and $N_0 = 125$ (bottom).}
\end{figure}

\newpage
\subsection{A code snippet}

Here's a snippet to illustrate the \textit{hp-adaptive} mesh refinement from the user's perspective:

\lstinputlisting[style=cpp, firstline=11]{../snippets/hp_refine.cpp}
\subsection{Smooth solutions}

Tests over a sequence of uniform meshes have been conducted to verify the algorithm's performance, confirming that:

\begin{gather} \label{trends}
    \lVert u - u^k_h \rVert_{\LT(\Omega)} \approx h^{k + 1}, \\
    \lVert u - u^k_h \rVert_{DG} \approx h^k.
\end{gather}

These results were obtained by selecting a smooth function as exact solution such as the following:

\begin{gather}
    u(x, y) = \sin(2 \pi x) \cos(2 \pi y),
\end{gather}

which leads to:

\begin{gather}
    f(x, y) = -\Delta u(x, y) = 8 \pi^2 \sin(2 \pi x) \cos(2 \pi y),
\end{gather}

and non-homogeneous Dirichlet boundary conditions given by the exact solution itself.

Error trends in the following pages.

\newpage
\subsubsection{Errors}

The following shows the error trends for the $\LT$ and $DG$ errors over sequences of uniform meshes over the square and L-shaped domains. The relations presented in \eqref{trends} have been confirmed.

% \begin{figure}[!ht]
% 	\centering

%     % TBA

% 	\caption{$\LT$ and DG errors versus mesh size on a sequence of uniform meshes over a square domain (top) and an L-shaped domain (bottom), $k = 2$ and $N \in \{100, 200, 400, 800\}$.}
% \end{figure}

\newpage
\subsection{Pathological solutions}

Tests over a sequence of uniform meshes, using pathological functions as exact solutions, highlight the need for an adaptive algorithm.

\cite{Antonietti2013} The pathological function for the square domain is:

\begin{gather} \label{pathological_square}
    u(x, y) = \frac{1 - e^{-100x}}{1 - e^{-100}} \sin(\pi y) (1 - x),
\end{gather}

which exhibits a strong boundary layer along the line $x = 0$.

For the L-shaped domain, the pathological function is:

\begin{gather} \label{pathological_lshape}
    u(\rho, \theta) = \rho^{2 / 3} \sin\left(\frac{2 \theta}{3}\right),
\end{gather}

for which we have $f = 0$ and we know that $u$ is analytical in $\Omega \setminus \Vector{0}$, but $\grad{u}$ is singular at the origin.

Error trends in the following pages.

\newpage
\subsubsection{Errors}

Given the pathological solution over the square, given its smoothness, the expected convergence rate is reached but at a slower rate.

% \begin{figure}[!ht]
% 	\centering
	
%     % TBA

% 	\caption{$\LT$ and DG errors versus mesh size on a sequence of uniform meshes over a square domain, $k = 3$, $N \in \{125, \dots, 16000\}$ (top) and $k = 4$, $N \in \{125, \dots, 8000\}$ (bottom).}
% \end{figure}

\newpage

Due to its nature, the pathological solution on the L-shaped domain does not reach the expected convergence rate.

% \begin{figure}[!ht]
% 	\centering
	
%     % TBA

% 	\caption{$\LT$ and DG errors versus mesh size on a sequence of uniform meshes over an L-shaped domain, $k = 3$ (top) and $k = 4$ (bottom). $N \in \{125, \dots, 8000\}$.}
% \end{figure}
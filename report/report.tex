\documentclass[12pt]{article}

% Bibliography.
\usepackage{cite}

\newcommand{\reporttitle}{Adaptive HP Discontinuous Galërkin Algorithms}
\newcommand{\accentcolor}{solarized-red}
\newcommand{\urlcolor}{solarized-yellow}

\usepackage{amsmath}
\usepackage{mathrsfs}
\usepackage{amsthm}
\usepackage{amsfonts}
\usepackage{bm}
\usepackage{amssymb}
\usepackage{stmaryrd}

% Sets and spaces.
\newcommand{\R}{\mathbb{R}}
\newcommand{\RT}{\mathbb{R}^2}
\newcommand{\N}{\mathbb{N}}

\newcommand{\PK}[1]{\mathbb{P}_{#1}}

\newcommand{\LT}{\mathscr{L}^2}
\newcommand{\HO}{\mathscr{H}^1}

\newcommand{\Tau}{\mathcal{T}}
\newcommand{\F}{\mathcal{F}}

% Vectors and operators.
\newcommand{\Vector}[1]{\bm{#1}}
\newcommand{\Operator}[1]{\textcolor{solarized-cyan}{#1}}

% Matrices.
\newcommand{\MA}{\mathcal{A}}
\newcommand{\VB}{\mathcal{B}}

% Gradient and divergence.
\newcommand{\grad}{\Vector{\nabla}}
\newcommand{\diver}{\text{div }}

% Span.
\newcommand{\Span}[1]{\text{span} \left\{ #1 \right\}}

% Bilinear operators.
\newcommand{\boa}[2]{\Operator{a}(#1, #2)}

% Redefinition.
\newcommand{\Exists}{\exists ~}
\newcommand{\Forall}{\forall ~}

% Theorems.
\newtheorem{theorem}{Theorem}[section]
\newtheorem{lemma}{Lemma}[section]
\newtheorem{proposition}{Proposition}[section]

\newtheorem*{theorem*}{Theorem}

\usepackage{courier}
\usepackage{listings}

\lstdefinestyle{default}{
	basicstyle=\ttfamily\color{solarized-base01},
	breakatwhitespace=false,
	breaklines=true,
	keepspaces=true,
	showspaces=false,
	showstringspaces=false,
	showtabs=false,
	tabsize=2
}

\lstdefinestyle{cpp}{ % C++
	commentstyle=\color{solarized-green},
	keywordstyle=\color{solarized-blue},
	stringstyle=\color{solarized-orange},
	basicstyle=\ttfamily\color{solarized-base02},
	numberstyle=\ttfamily\color{solarized-base01},
	breakatwhitespace=false,
	breaklines=true,
	captionpos=b,
	keepspaces=true,
	showspaces=false,
	language=c++,
	showstringspaces=false,
	showtabs=false,
	numbers=left,
	tabsize=4
}

\lstset{style=default}

\usepackage{nameref}
\usepackage{multicol}
\usepackage{titlesec}

\usepackage{enumerate}

\usepackage{graphicx}
\graphicspath{{gallery/}}

\usepackage{xcolor-solarized}
\color{solarized-base02}

\usepackage[T1]{fontenc}
\usepackage[utf8]{inputenc}

\usepackage[a4paper]{geometry}
\geometry{
	inner=20mm,
	outer=20mm,
	top=30mm,
	bottom=30mm,
	heightrounded,
	marginparwidth=50pt,
	marginparsep=20pt,
	headsep=25pt,
	headheight=30pt
}

\usepackage{hyperref}
\hypersetup{
	linktocpage=true,
	colorlinks=true,
	linkcolor=\accentcolor,
	urlcolor=\urlcolor,
	citecolor=\accentcolor,
	pdftitle={\reporttitle},
	pdfpagemode=FullScreen,
	pdfauthor={Andrea Di Antonio}
}

\usepackage{fancyhdr}
\pagestyle{fancy}
\fancyhf{}
\fancyhead[R]{Andrea Di Antonio}
\fancyhead[L]{\reporttitle}
\fancyfoot[C]{\thepage}

\title{\reporttitle}
\author{Andrea Di Antonio, 10655477} % Add supervisors.
\date{Exam session of July, 2024 \\ Academic Year 2023-24}

\setcounter{tocdepth}{2}

\begin{document}
	\pagenumbering{roman}
	\maketitle
	\thispagestyle{fancy}

	\begin{abstract}
		\begin{center}
			Report for the course \textit{Advanced Programming for Scientific Computing} on the implementation details of an adaptive HP discontinuous Galërkin algorithm.
		\end{center}
	\end{abstract}

	\newpage
	\tableofcontents

	\newpage
	\pagenumbering{arabic}
    \section{Formulation of the Poisson Problem}
	Consider the domain $\Omega \subset \mathbb{R}^2$. The aim is to find $u \in C^2(\Omega)$ such that, for any $f \in C(\Omega)$, the following holds:

\begin{gather}
    \begin{cases} \label{strong_stokes}
        - \Delta u = f & \Forall \Vector{x} \in \Omega, \\
        u = 0 & \Forall \Vector{x} \in \partial \Omega.
    \end{cases}
\end{gather}

\subsection{Weak formulation}

Now, seeking $u \in V$ such that, given $f \in V^*$, the following equation is satisfied:

\begin{gather} \label{weak_stokes}
    \boa{u}{v} = \langle f, v \rangle \quad \Forall v \in V,
\end{gather}

where $\boa{\cdot}{\cdot}: V \times V \rightarrow \mathbb{R}$ is defined as follows:

\begin{gather}
    \boa{u}{v} = \int_{\Omega} \grad u \cdot \grad v ~ d \omega. \label{a}
\end{gather}

	\newpage
    \section{Discontinuos Galërkin Method for the Poisson Problem}
	Consider now a symmetric interior penalty method for this problem.

For the DG formulation of the Poisson problem, the objective is to find $u^k_h \in V^k_h$ such that the following holds:

\begin{gather}
    \boa{u^k_h}{v^k_h} = \langle f, v^k_h \rangle \quad \Forall v^k_h \in V^k_h.
\end{gather}

Let $\left\{ \phi_i \right\}_{i = 1}^N$ denote a basis for the space $V^k_h$, we then have:

\begin{gather}
    u^k_h = \sum_{i = 1}^N \upsilon_i \phi_i \quad \Forall u^k_h \in V^k_h
\end{gather}

so that we aim to find $\Vector{\upsilon} \in \R^N$ such that:

\begin{gather}
    \MA \Vector{\upsilon} = \VB,
\end{gather}

where $\MA \in \R^{N \times N}$ and $\VB \in \R^N$ are defined as:

\begin{align}
    \MA_{ij} &= \boa{\phi_i}{\phi_j}, \label{matrix} \\ 
    \VB_i &= \langle f, \phi_i \rangle. \label{forcing}
\end{align}

The basis of choice will be that of Legendre polynomials.

\subsection{Discretization}

Let $\{\Tau_h\}_h$ be a sequence of polygonal meshes. Define $V^k_h$ as follows:

\begin{gather}
    V^k_h = \left\{ v^k_h \in \LT(\Omega): v^k_h \vert_K \in \PK{k}(K) ~ \Forall K \in \Tau_h \right\}.
\end{gather}

Hence we have the following formulation for $\boa{\cdot}{\cdot}$:

\begin{align} 
    \begin{split} \label{boa}
        \boa{v^k_h}{w^k_h} &= \sum_{K \in \Tau_h} \int_K \grad v^k_h \cdot \grad w^k_h ~ d \omega \\
        &- \sum_{F \in \F} \int_F \{\!\!\{ \grad w^k_h \}\!\!\} \cdot \llbracket v^k_h \rrbracket ~ d \sigma  \\
        &- \sum_{F \in \F} \int_F \llbracket w^k_h \rrbracket \cdot \{\!\!\{ \grad v^k_h \}\!\!\} ~ d \sigma \\
        &+ \sum_{F \in \F} \int_F \gamma \llbracket w^k_h \rrbracket \cdot \llbracket v^k_h \rrbracket ~ d \sigma \quad \Forall v^k_h, w^k_h \in V^k_h.
    \end{split}
\end{align}

\subsection{Non-homogenous Dirichlet boundary conditions}

Dirichlet boundary conditions can be enforced by penalization:

\begin{gather} \label{dirichlet}
    \llbracket u \rrbracket = (u - g) \Vector{n} \quad \Forall F \in \F : F \cap \partial \Omega \neq \emptyset.
\end{gather}

	\newpage
    \section{Polygonal Meshes Over a Polygonal Domain}
	\subsection{Building a mesh}

\begin{frame}
    \frametitle{Mesh-Building Strategy}

    The mesh-building strategy follows these steps:

    \begin{enumerate}
        \item Voronoi diagram generation,
        \item Diagram relaxation,
        \item Small edge collapse,
        \item Element connectivity analysis,
        \item Property evaluation.
    \end{enumerate}

    This process is facilitated by a thorough implementation of analytic geometry operations, including those involving lines and polygons.
\end{frame}

\begin{frame}[fragile]
    \frametitle{\lstinline{mesh_diagram}, \lstinline{mesh_relax}}

    Most steps of the mesh-building process are carried out by \lstinline{mesh_diagram}\footnote{Building and postprocessing.} and \lstinline{mesh_relax}.

    \begin{lstlisting}[style=cpp]
    std::vector<Polygon> mesh_diagram(
        const Polygon &, 
        const std::size_t &, 
        const bool &reflect = false, 
        const bool &uniform = false);

    std::vector<Polygon> mesh_relax(
        const Polygon &, 
        const std::vector<Polygon> &, 
        const bool &reflect = false);
    \end{lstlisting}

\end{frame}

\begin{frame}[fragile]
    \frametitle{\lstinline{Mesh}}

    \lstinline{struct Mesh} requires a polygonal domain, a diagram, and information on the elements' degrees.

    \begin{lstlisting}[style=cpp]
    Mesh(
        const Polygon &, 
        const std::vector<Polygon> &, 
        const std::vector<std::size_t> &);

    Mesh(
        const Polygon &, 
        const std::vector<Polygon> &, 
        const std::size_t &degree = 1);
    \end{lstlisting}

\end{frame}

\begin{frame}[fragile]
    \frametitle{\lstinline{Mesh} methods}

    The following methods are invoked by the mesh constructors to evaluate the diagram's properties.

    \begin{lstlisting}[style=cpp]
    std::vector<Element> mesh_elements(
        const std::vector<Polygon> &, 
        const std::vector<std::size_t> &);

    std::vector<std::vector<std::array<int, 3>>> 
    mesh_neighbours(
        const Polygon &, 
        const std::vector<Element> &);

    std::vector<Real> mesh_areas(
        const std::vector<Polygon> &);

    std::vector<Vector<Real>> mesh_max_simplices(
        const std::vector<Polygon> &);
    \end{lstlisting}

\end{frame}

\subsection{Examples}

\begin{frame}[fragile]
    \frametitle{A code snippet}

    The steps to create a mesh are schematized as follows:

    \begin{lstlisting}[style=cpp]
    Point a{0.0, 0.0};
    Point b{1.0, 0.0};
    Point c{1.0, 1.0};
    Point d{0.0, 1.0};

    Polygon domain{{a, b, c, d}};

    std::vector<Polygon> diagram = 
        mesh_diagram(domain, 100);
    
    Mesh mesh{domain, diagram};
    \end{lstlisting}

\end{frame}

\begin{frame}[fragile]
    \frametitle{A repository snippet}

    Building a mesh over a square or L-shaped domain is as simple as calling one of the two scripts provided with the repository.

    To create a mesh over a square domain with $N = 250$, simply compile the domain scripts by:

    \begin{lstlisting}
    make domains
    \end{lstlisting}

    and then use the \lstinline{square_domain} script by:

    \begin{lstlisting}
    ./executables/square_domain.out 250
    \end{lstlisting}

    Use \lstinline{polyplot} to show the newly created mesh:

    \begin{lstlisting}
    ./scripts/polyplot.py output/square_250.poly
    \end{lstlisting}

\end{frame}

	\newpage
	\bibliography{bibliography/refs.bib}
	\bibliographystyle{siam}

\end{document}

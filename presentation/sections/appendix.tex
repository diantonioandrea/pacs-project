% \subsection{The Poisson Problem and its Weak Formulation}

% \begin{frame}
%     \frametitle{The Poisson Problem}

%     Given a domain $\Omega \subset \mathbb{R}^2$, the Poisson problem is defined as finding \( u \in C^2(\Omega) \) such that, for any source function \( f \in C(\Omega) \) and boundary condition \( g \in C^2(\partial \Omega) \), the following holds:

%     \begin{gather}
%         \begin{cases} \label{strong_poisson}
%             - \Delta u = f & \Forall \Vector{x} \in \Omega, \\
%             u = g & \Forall \Vector{x} \in \partial \Omega.
%         \end{cases}
%     \end{gather}
% \end{frame}

% \begin{frame}
%     \frametitle{The Weak Formulation}

%     To find \( u \in \{v \in V : v \vert_{\partial \Omega} = g \} \), such that for any \( f \in V^* \), the following equation is satisfied:

%     \begin{gather}
%         \boa{u}{v} = \langle f, v \rangle \quad \text{for all } v \in V,
%     \end{gather}

%     where the bilinear form \( \boa{\cdot}{\cdot} \) is defined as:

%     \begin{gather}
%         \boa{u}{v} = \int_{\Omega} \grad u \cdot \grad v \, d\omega, \label{a}
%     \end{gather}

%     This weak formulation is the foundation for numerical approximation methods like the Discontinuous Galërkin (DG) method.
% \end{frame}

% \subsection{An Interior Penalty DG Method and a Polynomial Basis}

% \begin{frame}
%     \frametitle{Interior Penalty DG Method}

%     For solving the problem, we use a symmetric interior penalty method. The goal is to find \( u^k_h \in V^k_h \) such that:

%     \begin{gather}
%         \boa{u^k_h}{v^k_h} = \langle f, v^k_h \rangle \quad \Forall v^k_h \in V^k_h.
%     \end{gather}

%     The bilinear form \( \boa{\cdot}{\cdot} \) includes terms to account for discontinuities between elements, and is defined as:

%     \begin{align}
%         \begin{split}
%             \boa{v^k_h}{w^k_h} &= \sum_{K \in \Tau_h} \int_K \nabla v^k_h \cdot \nabla w^k_h \, d\omega - \sum_{F \in \F} \int_F \{\!\!\{ \nabla w^k_h \}\!\!\} \cdot \llbracket v^k_h \rrbracket \, d\sigma \\
%             &- \sum_{F \in \F} \int_F \llbracket w^k_h \rrbracket \cdot \{\!\!\{ \nabla v^k_h \}\!\!\} \, d\sigma + \sum_{F \in \F} \int_F \gamma \llbracket w^k_h \rrbracket \cdot \llbracket v^k_h \rrbracket \, d\sigma \\
%             &= \Operator{v}(v^k_h, w^k_h) - \Operator{i}(v^k_h, w^k_h) - \Operator{i}(w^k_h, v^k_h) + \Operator{s}(v^k_h, w^k_h).
%         \end{split}
%     \end{align}
% \end{frame}

% \begin{frame}
%     \frametitle{Polynomial Basis}

%     Let \( \{ \phi_i \}_{i = 1}^N \) denote a basis for the space \( V^k_h \). The goal is to find \( \mathbf{\upsilon} \in \mathbb{R}^N \) such that:

%     \begin{gather}
%         \MA \mathbf{\upsilon} = \VB,
%     \end{gather}
    
%     where \( \MA \in \mathbb{R}^{N \times N} \) and \( \VB \in \mathbb{R}^N \) are defined as:

%     \begin{align}
%         \MA_{ij} &= \boa{\phi_i}{\phi_j}, \\
%         \VB_i &= \langle f, \phi_i \rangle.
%     \end{align}

%     Legendre polynomials are used as the basis functions for \( V^k_h \) due to their orthogonality and numerical properties.
% \end{frame}



% \subsection{\textit{h-Adaptivity}}

% \begin{frame}
%     \frametitle{\textit{h-Adaptivity}}

%     The need for \textit{h-adaptivity} arises from the inefficiency encountered when solving the Poisson problem over sequences of uniform meshes, especially when dealing with low-regularity exact solutions.

%     To implement \textit{h-adaptivity}, the first step is to evaluate the $\LT$ error on each element and then refine the element with the highest error according to a specific refinement strategy.    

% \end{frame}

% \begin{frame}
%     \frametitle{Refinement strategy}

%     \begin{enumerate}
%         \item For polygons with $N_e \leq 4$, the refiner adds a single node at the polygon's centroid.
%         \item For polygons with $N_e > 4$, the refiner adds $N_e$ new nodes at the midpoints of the segments connecting the polygon's centroid to the midpoints of its edges.
%     \end{enumerate}

%     The elements $K$ to be refined are chosen in the following way:

%     \begin{gather}
%         \eta_K > \sigma \eta_{M},
%     \end{gather}

% \end{frame}

% \begin{frame}
%     \frametitle{\textit{A posteriori} error estimator}

%     The second step in implementing \textit{h-adaptivity} is to define an \textit{a posteriori} error estimator, which enables the identification of elements that need refinement without requiring any information about the exact solution.

%     One possible approach considers the following upper bound on the error:

%     \begin{gather}
%         \lVert u - u^k_h \rVert_{\LT(\Omega)} \leq C_{ub} \sum_{K \in \Tau_h} (R_K^2 + O_K^2),
%     \end{gather}

%     where $R_K^2 = R_{K, E}^2 + R_{K, N}^2 + R_{K, J}^2 + R_{K, T}^2$ is the local estimator and $O_K^2 = O_{K, E}^2 + O_{K, J}^2 + O_{K, T}^2$ is the local data oscillation.

% \end{frame}

% \begin{frame}
%     \frametitle{\textit{A posteriori} error estimator}

%     Each term is given by:

%     \begin{align}
%         R_{K, E} &= \lVert h (\bar{f} + \Delta u^k_h) \rVert_{\LT(K)},\notag \\
%         R_{K, N} &= \lVert h^{1/2} \llbracket \grad u^k_h \cdot \Vector{n} \rrbracket \rVert_{\LT(\partial K)},\notag \\
%         R^2_{K, J} &= \lVert \gamma^{1/2} \llbracket u^k_h \rrbracket \rVert^2_{\LT(\partial K \cap \Gamma_{i})} + \lVert \gamma^{1/2} (u^k_h - \bar{g}) \rVert^2_{\LT(\partial K \cap \partial \Omega)},\notag \\
%         R^2_{K, T} &= \lVert h^{1/2} \llbracket \grad u^k_h \cdot \Vector{e} \rrbracket \rVert^2_{\LT(\partial K \cap \Gamma_{i})} + \lVert \gamma^{1/2} \grad (u^k_h - \bar{g}) \cdot \Vector{e} \rVert^2_{\LT(\partial K \cap \partial \Omega)},\notag \\
%         O_{K, E} &= \lVert h (f - \bar{f}) \rVert_{\LT(K)},\notag \\
%         O_{K, J} &= \lVert \gamma^{1/2} (g - \bar{g}) \rVert_{\LT(\partial K \cap \partial \Omega)},\notag \\
%         O_{K, T} &= \lVert h^{1/2} \grad (g - \bar{g}) \cdot \Vector{e} \rVert_{\LT(\partial K \cap \partial \Omega)}. \notag
%     \end{align}

% \end{frame}

% \subsection{\textit{hp-Adaptivity}}

% \begin{frame}
%     \frametitle{\textit{hp-Adaptivity}}

%     The next and final step is to implement \textit{p-adaptive} refinement using a test of analyticity.

%     Since the solution is represented by the coefficients of Legendre polynomials, analyticity can be assessed by evaluating their rate of decay. Assuming smoothness for $u^k_{h, K}$, the following holds:

%     \begin{gather}
%         \Exists a_K, b_K \in \R : c_{ij} \approx a_K e^{-b_K (i + j)}.
%     \end{gather}

%     The elements $K$ to be refined are chosen in the following way:

%     \begin{gather}
%         \eta_K^2 > \sigma \bar{\eta}^2,
%     \end{gather}

%     The value of $b_K$ lets us decide whether to \textit{h-refine} or \textit{p-refine} an element.

% \end{frame}



\subsection{Quadrature}

\begin{frame}[fragile]
    \frametitle{Quadrature}

    Quadratures are performed using the Gauss-Legendre quadrature rule, with quadrature nodes and weights generated by the \lstinline{gauss_legendre} function.

    \begin{lstlisting}[style=cpp]
    std::array<Vector<Real>, 2> gauss_legendre(
        const Real &, 
        const Real &, 
        const std::size_t &);
    \end{lstlisting}
\end{frame}

\begin{frame}[fragile]
    \frametitle{Quadrature}

    Quadrature nodes and weights are then adapted to the reference interval and square by \lstinline{quadrature_1d} and \lstinline{quadrature_2d}.

    \begin{lstlisting}[style=cpp]
    std::array<Vector<Real>, 2> quadrature_1d(
        const std::size_t &);
    std::array<Vector<Real>, 3> quadrature_2d(
        const std::size_t &);
    \end{lstlisting}
\end{frame}
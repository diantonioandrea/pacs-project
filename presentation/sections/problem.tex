\subsection{The Poisson Problem and Its Weak Formulation}

\begin{frame}
    \frametitle{The Poisson Problem}

    Given a domain $\Omega \subset \mathbb{R}^2$, the Poisson problem is defined as finding \( u \in C^2(\Omega) \) such that, for any source function \( f \in C(\Omega) \) and boundary condition \( g \in C^2(\partial \Omega) \), the following holds:

    \begin{gather}
        \begin{cases} \label{strong_poisson}
            - \Delta u = f & \Forall \Vector{x} \in \Omega, \\
            u = g & \Forall \Vector{x} \in \partial \Omega.
        \end{cases}
    \end{gather}
\end{frame}

\begin{frame}
    \frametitle{The Weak Formulation}

    To find \( u \in \{v \in V : v \vert_{\partial \Omega} = g \} \), such that for any \( f \in V^* \), the following equation is satisfied:

    \begin{gather}
        \boa{u}{v} = \langle f, v \rangle \quad \text{for all } v \in V,
    \end{gather}

    where the bilinear form \( \boa{\cdot}{\cdot} \) is defined as:

    \begin{gather}
        \boa{u}{v} = \int_{\Omega} \grad u \cdot \grad v \, d\omega, \label{a}
    \end{gather}

    This weak formulation is the foundation for numerical approximation methods like the Discontinuous Galërkin (DG) method.
\end{frame}

\subsection{An Interior Penalty DG Method and a Polynomial Basis}

\begin{frame}
    \frametitle{Interior Penalty DG Method}

    For solving the problem, we use a symmetric interior penalty method. The goal is to find \( u^k_h \in V^k_h \) such that:

    \begin{gather}
        \boa{u^k_h}{v^k_h} = \langle f, v^k_h \rangle \quad \Forall v^k_h \in V^k_h.
    \end{gather}

    The bilinear form \( \boa{\cdot}{\cdot} \) includes terms to account for discontinuities between elements, and is defined as:

    \begin{align}
        \begin{split}
            \boa{v^k_h}{w^k_h} &= \sum_{K \in \Tau_h} \int_K \nabla v^k_h \cdot \nabla w^k_h \, d\omega - \sum_{F \in \F} \int_F \{\!\!\{ \nabla w^k_h \}\!\!\} \cdot \llbracket v^k_h \rrbracket \, d\sigma \\
            &- \sum_{F \in \F} \int_F \llbracket w^k_h \rrbracket \cdot \{\!\!\{ \nabla v^k_h \}\!\!\} \, d\sigma + \sum_{F \in \F} \int_F \gamma \llbracket w^k_h \rrbracket \cdot \llbracket v^k_h \rrbracket \, d\sigma.
        \end{split}
    \end{align}

    where \(\gamma\) is the penalty parameter ensuring stability.
\end{frame}

\begin{frame}
    \frametitle{Polynomial Basis}

    Let \( \{ \phi_i \}_{i = 1}^N \) denote a basis for the space \( V^k_h \). The goal is to find \( \mathbf{\upsilon} \in \mathbb{R}^N \) such that:

    \begin{gather}
        \MA \mathbf{\upsilon} = \VB,
    \end{gather}
    
    where \( \MA \in \mathbb{R}^{N \times N} \) and \( \VB \in \mathbb{R}^N \) are defined as:

    \begin{align}
        \MA_{ij} &= \boa{\phi_i}{\phi_j}, \\
        \VB_i &= \langle f, \phi_i \rangle.
    \end{align}

    Legendre polynomials are used as the basis functions for \( V^k_h \) due to their orthogonality and numerical properties.
\end{frame}
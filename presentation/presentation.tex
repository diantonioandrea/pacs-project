\documentclass[10pt]{beamer}

\usepackage[T1]{fontenc}
\usepackage[utf8]{inputenc}

% Bibliography.
\usepackage[backend=biber, style=ieee]{biblatex}
\addbibresource{../report/bibliography/refs.bib}

\usepackage{xcolor-solarized}
\color{solarized-base02}

% Beamer.
\usetheme{metropolis}
\useinnertheme{circles}

\usecolortheme{seahorse} % Double color theme.
\usecolortheme[named=solarized-blue]{structure}
\setbeamercolor{background canvas}{bg=white}

\AtBeginSection[]
{
  \begin{frame}{ToC}
    \frametitle{Contents}
    \tableofcontents[sectionstyle=show/hide,subsectionstyle=show/show/hide]
  \end{frame}
}

\newcommand{\presentationtitle}{The \textit{hp-Adaptive} Discontinuous Galërkin Method}
\newcommand{\accentcolor}{solarized-blue}
\newcommand{\urlcolor}{\accentcolor}

\usepackage{amsmath}
\usepackage{mathrsfs}
\usepackage{amsthm}
\usepackage{amsfonts}
\usepackage{bm}
\usepackage{amssymb}
\usepackage{stmaryrd}

% Sets and spaces.
\newcommand{\R}{\mathbb{R}}
\newcommand{\RT}{\mathbb{R}^2}
\newcommand{\N}{\mathbb{N}}

\newcommand{\PK}[1]{\mathbb{P}_{#1}}

\newcommand{\LT}{\mathscr{L}^2}
\newcommand{\HO}{\mathscr{H}^1}

\newcommand{\Tau}{\mathcal{T}}
\newcommand{\F}{\mathcal{F}}

% Vectors and operators.
\newcommand{\Vector}[1]{\bm{#1}}
\newcommand{\Operator}[1]{\textcolor{solarized-cyan}{#1}}

% Matrices.
\newcommand{\MM}{\mathcal{M}}
\newcommand{\MA}{\mathcal{A}}
\newcommand{\VB}{\mathcal{B}}

% Gradient and divergence.
\newcommand{\grad}{\Vector{\nabla}}
\newcommand{\diver}{\text{div }}

% Span.
\newcommand{\Span}[1]{\text{span} \left\{ #1 \right\}}

% Bilinear operators.
\newcommand{\boa}[2]{\Operator{a}(#1, #2)}

% Redefinition.
\newcommand{\Exists}{\exists ~}
\newcommand{\Forall}{\forall ~}

% % Theorems.
% \newtheorem{theorem}{Theorem}[section]
% \newtheorem{lemma}{Lemma}[section]
% \newtheorem{proposition}{Proposition}[section]

% \newtheorem*{theorem*}{Theorem}

\usepackage{courier}
\usepackage{listings}

\lstdefinestyle{default}{
	basicstyle=\ttfamily\color{solarized-base01},
	breakatwhitespace=false,
	breaklines=true,
	keepspaces=true,
	showspaces=false,
	showstringspaces=false,
	showtabs=false,
	tabsize=2
}

\lstdefinestyle{cpp}{ % C++
	commentstyle=\color{solarized-green},
	keywordstyle=\color{solarized-blue},
	stringstyle=\color{solarized-orange},
	basicstyle=\ttfamily\color{solarized-base02},
	numberstyle=\ttfamily\color{solarized-base01},
	breakatwhitespace=false,
	breaklines=true,
	captionpos=b,
	keepspaces=true,
	showspaces=false,
	language=c++,
	showstringspaces=false,
	showtabs=false,
	tabsize=4
}

\lstset{style=default}

\usepackage{graphicx}
\graphicspath{{../report/gallery/}}

% Plots.
\usepackage{subcaption}

% TikZ.
\usepackage{tikz}
\usepackage{pgfplots}
\pgfplotsset{compat=1.18}

\usepackage{hyperref}
\hypersetup{
	pdftitle={\presentationtitle},
	pdfpagemode=FullScreen,
	pdfauthor={Andrea Di Antonio}
}

\title{\presentationtitle}
\subtitle{Advanced Programming for Scientific Computing}
\author[Andrea Di Antonio]{Andrea Di Antonio \\ Supervised by Professors Paola F. Antonietti and Marco Verani} % I should NOT use "\\" here.
\date{Exam session of September 10, 2024 \\ Academic Year 2023-24}

\begin{document}

    \begin{frame}
        \titlepage
    \end{frame} % Warning on title.

	\begin{frame}{ToC}
		\frametitle{Contents}
		\tableofcontents[hideallsubsections]
	\end{frame}

    \section{Introduction}
    Consider the domain $\Omega \subset \mathbb{R}^2$. The aim is to find $u \in C^2(\Omega)$ such that, for any $f \in C(\Omega)$, the following holds:

\begin{gather}
    \begin{cases} \label{strong_stokes}
        - \Delta u = f & \Forall \Vector{x} \in \Omega, \\
        u = 0 & \Forall \Vector{x} \in \partial \Omega.
    \end{cases}
\end{gather}

\subsection{Weak formulation}

Now, seeking $u \in V$ such that, given $f \in V^*$, the following equation is satisfied:

\begin{gather} \label{weak_stokes}
    \boa{u}{v} = \langle f, v \rangle \quad \Forall v \in V,
\end{gather}

where $\boa{\cdot}{\cdot}: V \times V \rightarrow \mathbb{R}$ is defined as follows:

\begin{gather}
    \boa{u}{v} = \int_{\Omega} \grad u \cdot \grad v ~ d \omega. \label{a}
\end{gather}

	\section{Polygonal Meshes}
	\subsection{Building a mesh}

\begin{frame}
    \frametitle{Mesh-Building Strategy}

    The mesh-building strategy follows these steps:

    \begin{enumerate}
        \item Voronoi diagram generation,
        \item Diagram relaxation,
        \item Small edge collapse,
        \item Element connectivity analysis,
        \item Property evaluation.
    \end{enumerate}

    This process is facilitated by a thorough implementation of analytic geometry operations, including those involving lines and polygons.
\end{frame}

\begin{frame}[fragile]
    \frametitle{\lstinline{mesh_diagram}, \lstinline{mesh_relax}}

    Most steps of the mesh-building process are carried out by \lstinline{mesh_diagram}\footnote{Building and postprocessing.} and \lstinline{mesh_relax}.

    \begin{lstlisting}[style=cpp]
    std::vector<Polygon> mesh_diagram(
        const Polygon &, 
        const std::size_t &, 
        const bool &reflect = false, 
        const bool &uniform = false);

    std::vector<Polygon> mesh_relax(
        const Polygon &, 
        const std::vector<Polygon> &, 
        const bool &reflect = false);
    \end{lstlisting}

\end{frame}

\begin{frame}[fragile]
    \frametitle{\lstinline{Mesh}}

    \lstinline{struct Mesh} requires a polygonal domain, a diagram, and information on the elements' degrees.

    \begin{lstlisting}[style=cpp]
    Mesh(
        const Polygon &, 
        const std::vector<Polygon> &, 
        const std::vector<std::size_t> &);

    Mesh(
        const Polygon &, 
        const std::vector<Polygon> &, 
        const std::size_t &degree = 1);
    \end{lstlisting}

\end{frame}

\begin{frame}[fragile]
    \frametitle{\lstinline{Mesh} methods}

    The following methods are invoked by the mesh constructors to evaluate the diagram's properties.

    \begin{lstlisting}[style=cpp]
    std::vector<Element> mesh_elements(
        const std::vector<Polygon> &, 
        const std::vector<std::size_t> &);

    std::vector<std::vector<std::array<int, 3>>> 
    mesh_neighbours(
        const Polygon &, 
        const std::vector<Element> &);

    std::vector<Real> mesh_areas(
        const std::vector<Polygon> &);

    std::vector<Vector<Real>> mesh_max_simplices(
        const std::vector<Polygon> &);
    \end{lstlisting}

\end{frame}

\subsection{Examples}

\begin{frame}[fragile]
    \frametitle{A code snippet}

    The steps to create a mesh are schematized as follows:

    \begin{lstlisting}[style=cpp]
    Point a{0.0, 0.0};
    Point b{1.0, 0.0};
    Point c{1.0, 1.0};
    Point d{0.0, 1.0};

    Polygon domain{{a, b, c, d}};

    std::vector<Polygon> diagram = 
        mesh_diagram(domain, 100);
    
    Mesh mesh{domain, diagram};
    \end{lstlisting}

\end{frame}

\begin{frame}[fragile]
    \frametitle{A repository snippet}

    Building a mesh over a square or L-shaped domain is as simple as calling one of the two scripts provided with the repository.

    To create a mesh over a square domain with $N = 250$, simply compile the domain scripts by:

    \begin{lstlisting}
    make domains
    \end{lstlisting}

    and then use the \lstinline{square_domain} script by:

    \begin{lstlisting}
    ./executables/square_domain.out 250
    \end{lstlisting}

    Use \lstinline{polyplot} to show the newly created mesh:

    \begin{lstlisting}
    ./scripts/polyplot.py output/square_250.poly
    \end{lstlisting}

\end{frame}

	\section{Solving the Poisson Problem}
	Having built a mesh over a polygonal domain, the Poisson problem can be solved by first constructing the problem's matrix $\MA$ and the forcing term $\VB$ as shown in \eqref{matrix} and \eqref{forcing}.

The \lstinline{laplacian} function constructs the matrices used for solving the problem and evaluating errors by computing all terms in \eqref{boa} for each element. The resulting matrices are in sparse form.

The \lstinline{forcing} function constructs the forcing term by evaluating \eqref{forcing} and enforces the Dirichlet boundary condition \eqref{dirichlet} through penalization.

\cite{Saad2003} The solution to the matrix equation $\MA \Vector{\upsilon}^k_h = \VB$ is obtained using the \lstinline{BICGSTAB} algorithm, with the \lstinline{GMRES} algorithm used if the first one fails to converge within a fixed number of steps. Both of these iterative algorithms for sparse matrices have been implemented in the \textit{algebra} section of the code.

Let $\MM$ and $\MA_{DG}$ be the mass and $DG$ matrices. The $L^2$ and $DG$ errors are then evaluated by first computing the modal coefficients $\Vector{\upsilon}_m$ for the exact solution $u$ and solving $\MM \Vector{u} = \Vector{\upsilon}_m$ using the \lstinline{CG} algorithm. Thus, the error vector is $\Vector{e} = \Vector{\upsilon} - \Vector{\upsilon}^k_h$. Hence:

\begin{gather}
    \lVert u - u^k_h \rVert_{\LT(\Omega)} = \sqrt{\Vector{e}^\intercal \MM \Vector{e}}, \\
    \lVert u - u^k_h \rVert_{DG} = \sqrt{\Vector{e}^\intercal \MA_{DG} \Vector{e}}.
\end{gather}

Some examples in the following sections.

\newpage
\subsection{A code snippet}

Here's a snippet to illustrate the Poisson solution process from the user's perspective:

\lstinputlisting[style=cpp, firstline=11]{../snippets/poisson.cpp}

	\section{Implementing \textit{h-Adaptivity} and \textit{hp-Adaptivity}}
	\subsection{\textit{h-Adaptivity}}

\begin{frame}[fragile]
    \frametitle{\textit{h-Adaptivity}}

    \begin{description}
        \item[Error evaluation] To implement \textit{h-adaptivity}, the first step is to evaluate the error on each element, initially using the exact error, and then applying an \textit{a-posteriori} error estimator.
        \item[Marking] The elements $K$ to be refined are then selected as follows:
            \begin{gather}
                \eta_K > \sigma \eta_{M},
            \end{gather}
        \item[Refinement] The function \lstinline{mesh_refine_size} refines the mesh elements based on the local errors according to a specific refinement strategy.
        \begin{lstlisting}[style=cpp]
        void mesh_refine_size(
            Mesh &, 
            const Mask &);
        \end{lstlisting}
    \end{description}

\end{frame}

\subsection{\textit{hp-Adaptivity}}

\begin{frame}[fragile]
    \frametitle{\textit{hp-Adaptivity}}

    \begin{description}
        \item[Error evaluation] The next and final step is to implement \textit{p-adaptive} refinement using a test of analyticity. Analyticity can be assessed by evaluating the rate of decay of Legendre coefficients. Assuming smoothness for $u^k_{h, K}$, the following relationship holds:
        \begin{gather}
            \Exists a_K, b_K \in \R : c_{ij} \approx a_K e^{-b_K (i + j)}.
        \end{gather}
        \item[Marking] The elements $K$ to be refined are selected as follows:
        \begin{gather}
            \eta_K^2 > \sigma \bar{\eta}^2,
        \end{gather}
    \end{description}

\end{frame}

\begin{frame}[fragile]
    \frametitle{\textit{hp-Adaptivity}}

    \begin{description}
        \item[Refinement] The function \lstinline{mesh_refine} refines the mesh elements based on local errors and analyticity, choosing between \textit{h-refinement} and \textit{p-refinement}.

    \begin{lstlisting}[style=cpp]
    void mesh_refine(
        Mesh &, 
        const Estimator &, 
        const Real &refine = 0.75, 
        const Real &speed = 1.0);

    void mesh_refine_degree(
        Mesh &, 
        const Mask &);
    \end{lstlisting}
    \end{description}

\end{frame}

\subsection{Estimates}

\begin{frame}[fragile]
    \frametitle{Estimates}

    Estimates are computed using the \lstinline{Estimators} class, instantiated as follows:

    \begin{lstlisting}[style=cpp]
    Estimator(
        const Mesh &, 
        const Sparse<Real> &, 
        const Vector<Real> &, 
        const Functor &, 
        const Functor &dirichlet
            = Functor{}, 
        const TwoFunctor &dirichlet_gradient
            = TwoFunctor{}, 
        const Real &penalty_coefficient = 10.0);
    \end{lstlisting}

\end{frame}

\subsection{Code examples}

\begin{frame}[fragile]
    \frametitle{A code snippet}

    The steps to \textit{hp-refine} a mesh are outlined as follows:

    \begin{lstlisting}[style=cpp]
    [...]

    auto [M, A, DGA] = laplacian(mesh);
    Vector<Real> B = forcing(mesh, source);

    Vector<Real> numerical = lapsolver(mesh, A, B);
    Estimator est{mesh, M, numerical, source};

    mesh_refine(mesh, est);
    \end{lstlisting}

\end{frame}

	% \section{Numerical Tests}

	% References.
	\begin{frame}[allowframebreaks]
		\nocite{*}
		\printbibliography
	\end{frame}

	% Compiled on.
	\begin{frame}
		\thispagestyle{empty}
		\begin{center}
			Compiled on \today.
		\end{center}
	\end{frame}

\end{document}